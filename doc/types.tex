
\chapter{Datatypes and schemas}

This chapter presents two closely related packages: {\tt itools.datatypes}
and {\tt itools.schema}.

\subsubsection{Introduction}

Information stored in a file or sent through a network appears as a chain
of bytes, but it represents high level information. For example we may
want to keep some basic data about a master piece book:


\begin{code}
  <metadata xmlns:dc="http://purl.org/dc/elements/1.1/">
    <dc:title>100 a�os de soledad</dc:title>
    <dc:creator>Gabriel Garcia Marquez</dc:creator>
    <dc:date>1999-07-02</dc:date>
    <dc:language>es</dc:language>
  </metadata>
\end{code}

Above we have used the XML language to keep the title, author, date of
publication and language of our book; but we may prefer a different
file format:

\begin{code}
  BEGIN:RECORD
    TITLE:100 a�os de soledad
    CREATOR:Gabriel Garcia Marquez
    DATE:1999-07-02
    LANGUAGE:es
  END:RECORD
\end{code}

Indepently of the format used, one thing is sure, we would like to load the
information with a type:

\begin{code}
  '100 a�os de soledad'     -(unicode)->  u'100 a�os de soledad'
  'Gabriel Garcia Marquez'  -(unicode)->  u'Gabriel Garcia Marquez'
  '1999-07-02'              ----(date)->  datetime.date(1999, 7, 2)
  'es'                      --(string)->  'es'
\end{code}

This process called {\em deserialization}. The opposite operation, which
we are also interested in, is called {\em serialization}: to transform a
value to a chain of bytes.

Other operations we may be interested include:

\begin{itemize}
  \item To check wether a string value is correct; for example, to check
    wether the date respects the ISO 8601 standard.

  \item To provide a default value when a field is missing; for example we
    may assume the language is English if not specified otherwise.
\end{itemize}

The packages {\tt itools.datatypes} and {\tt itools.schema} provide an
infrastructure to do all this and more. What is useful not only to serialize
and deserialize files, but also for other purposes, like validating user
input data.


\section{Datatypes}

In Python there are basic types like {\tt unicode}, {\tt integer} or
{\tt float}. And the there are more complex types like dates.

The module {\tt itools.datatypes} provides an infrastructure orthogonal to
the Python types. The basic service provided by this infrastructure is the
deserialization and serialization of values; which is implemented as the
couple of class methods {\tt decode} and {\tt encode}, for example:

\begin{code}
    >>> from itools.datatypes import DateTime
    >>> datetime = DateTime.decode('2005-05-02 16:47')
    >>> datetime
    datetime.datetime(2005, 5, 2, 16, 47)
    >>> DateTime.encode(datetime)
    '2005-05-02 16:47'
\end{code}

This approach, to implement the serialization/deserialization code separate
from the type itself, allows to avoid subclassig built-in types, what has a
performance impact.

This also illustrates one of the software principles behind the itools coding,
different programming aspects should be cleary distinct in the implementation
and programming interface.


\subsection{Out of the box}

Out-of-the-box {\tt itools.datatypes} provides support for the following
types:

\begin{description}
  \item [Integers ({\tt Integer})]
    An integer number is serialized using ASCII characters. This means a
    call to {\tt Integer.decode(x)} is equivalent to {\tt int(x)}, and
    {\tt Integer.encode(x)} does the same than {\tt str(x)}.

  \item [Text ({\tt Unicode})]
    Text strings are serialized using the UTF-8 encoding (by default).

  \item [Byte strings ({\tt String})]
    A byte string does not needs to be serialized or deserialized, the
    output is always equal to the input.

  \item [Booleans ({\tt Boolean})]
     Boolean values are encoded with the ``0'' character for the {\em false}
     value and with the ``1'' character for the {\em true} value.

  \item [Dates ({\tt Date})]
    Dates are encoded following the ISO 8601 standard\footnote{http://www.iso.org/iso/en/prods-services/popstds/datesandtime.html}: {\tt YYYY-MM-DD}.

  \item [Datetimes ({\tt DateTime})]
    Date and time is encoded with the pattern: {\tt YYYY-MM-DD hh:mm}.

  \item [URIs ({\tt URI})]
    The URI decoder will build and return one of the URI reference objects
    defined in the {\tt itools.uri} package, usually it will be an instance
    of the class {\tt itools.uri.generic.Reference}.

  \item [Filenames ({\tt FileName})]
    Usually filenames include extensions to indicate the file type, and
    sometimes other information like the language. The filename decoder
    will parse a filename and return a tuple where the first element is
    the filename, the second element is the file type, and the last element
    is the language. For example:

\begin{code}
    >>> from itools.datatypes import FileName
    >>> FileName.decode('index.html.en')
    ('index', 'html', 'en')
    >>> FileName.decode('index.html')
    ('index', 'html', None)
    >>> FileName.decode('index')
    ('index', None, None)
\end{code}

  \item [XML qualified names ({\tt QName})]
    An XML qualified name has two parts, the prefix and the local name, so
    our decoder will return a tuple with these two elements:

\begin{code}
    >>> from itools.datatypes import QName
    >>> QName.decode('dc:title')
    ('dc', 'title')
    >>> QName.decode('href')
    (None, 'href')
\end{code}

    The encoder expects a two element tuple:

\begin{code}
    >>> QName.encode(('dc', 'title'))
    'dc:title'
    >>> QName.encode((None, 'href'))
    'href'
\end{code}

\end{description}


\subsection{Defining new datatypes}

There are two ways to define a new datatype: subclassing and instanciating.

\subsubsection{Subclassing}

All datatytpes inherit from the abstract class {\tt DataType}. To define a
new datatype either subclass directly from {\tt DateType}, or from any
other subclass of it.

As an example we are going to define a datatype that loads mimetypes as
a two elements tuple:

\begin{code}
  import mimetypes
  from itools.datatypes import DateType

  class MimeType(DateType):

      default = None

      @staticmethod
      def decode(data):
          return tuple(data.split('/'))

      @staticmethod
      def encode(value)
          return '%s/%s' % value

      @staticmethod
      def is_valid(data):
          return mimetypes.guess_extension(data) is not None
\end{code}

Two things to highligth:

\begin{itemize}
  \item We have set the default value to {\tt None}, though this is
    not really needed since the {\tt DataType} class already defines this
    variable to {\tt None}.

    Another good default value maybe {\tt ('application', 'octet-stream')}.

  \item We have added the method {\tt is\_valid}, which is not defined by
    any other datatype included in {\tt itools}. This illustrates that the 
    datatypes can be extended with whatever logic, which we could use
    later in the application code.
\end{itemize}


\subsection{Instanciating}

This is a more compact way to specialize a datatype, when the changes are
small. For example:

\begin{code}
  from itools.datatypes import String

  WorkflowState = String(default='private')
\end{code}

Here we have defined a workflow state as an string, whose default value is
{\tt private}.

Note that a shortcoming of this approach is that, unlike subclassing, it is
not possible to instanciate a datatype that already is an instance.


\section{Schemas}