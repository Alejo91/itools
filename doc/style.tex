\chapter{Coding style guide}

This chapter describes the coding conventions used to write {\tt itools}.
If you ever contribute a patch to {\tt itools}, be sure your code adheres
to these rules.

Sometimes this guide contradicts the {\em PEP
8}\footnote{http://www.python.org/peps/pep-0008.html} recommendations; in
these cases this guide applies, in the context of {\tt itools}.


\section{Language and encoding}

Code must be written in english. The preferred encoding is {\em UTF-8},
though others encodings are allowed.


\section{Module structure}

Each module is splitted in six sections:

\begin{enumerate}
  \item encoding statement;
  \item copyright notice;
  \item license reference;
  \item the module documentation string;
  \item import statements;
  \item the code itself.
\end{enumerate}

Here we are going to describe the module's header, made up of the first
five sections. The style for the code itself will described in the rest
of this chapter.


\subsubsection{The encoding}

It is only requiered if it is any other encoding than ASCII. Example:

\begin{code}
    # -*- coding: UTF-8 -*-
\end{code}

\subsubsection{The Copyright}

After the encoding comes the copyright, whose structure is:

\begin{code}
    # Copyright (C) <years> <author name> <email>
                    <years> <author name> <email>
                    ...
\end{code}

\subsubsection{The License}

Right after the copyright statement comes a reference to the license. For
{\tt itools} it is the {\em LGPL}.

\subsubsection{The module's documentation string}

There should be a documentation string explaining what the module does
(though it is better to have none than a dummy one).

The explanation about how to write a documentation string is out of the
scope of this chapter, it is covered by the {\em PEP
257}\footnote{http://www.python.org/peps/pep-0257.html}.


\subsubsection{Imports}

The imports statements are at the top of the file (just after the legal
statements and the module docstrings), though there may be imports within
a function or method to avoid circular references.

Imports should be grouped, with the order being:

\begin{enumerate}
  \item standard library imports
  \item {\tt itools} imports
  \item other package package imports
\end{enumerate}

Every group should be preceded by an introductory comment, like:

\begin{code}
    # Import from the Standard Library
    ...

    # Import from itools
    ...
\end{code}


\subsection{Example}

For example, at the time of this writting the module
{\tt itools.handlers.Handler} starts by:

\begin{code}
    # -*- coding: ISO-8859-1 -*-
    # Copyright (C) 2003-2004 Juan David Ib��ez Palomar <jdavid@itaapy.com>
    #
    # This library is free software; you can redistribute it and/or
    # modify it under the terms of the GNU Lesser General Public
    # License as published by the Free Software Foundation; either
    # version 2.1 of the License, or (at your option) any later version.
    #
    # This library is distributed in the hope that it will be useful,
    # but WITHOUT ANY WARRANTY; without even the implied warranty of
    # MERCHANTABILITY or FITNESS FOR A PARTICULAR PURPOSE.  See the GNU
    # Lesser General Public License for more details.
    #
    # You should have received a copy of the GNU Lesser General Public
    # License along with this library; if not, write to the Free Software
    # Foundation, Inc., 59 Temple Place, Suite 330, Boston, MA 02111-1307  USA

    """
    This module provides the abstract class which is the root in the
    handler class hierarchy.
    """

    # Import from the Standard Library
    import datetime

    # Import from itools
    from itools import uri
    from itools.resources import base
\end{code}



\section{Format rules}

\subsubsection{Indentation: 4 spaces}

Each indentation level must have four (4) spaces. Never use tabs.

\subsubsection{Maximum line length: 80 characters}

Lines should be 80 characters wide at most.

The preferred way of wrapping long lines is by using Python's implied line
continuation inside parentheses, brackets and braces. If necessary, you can
add an extra pair of parentheses around an expression, but usually using
a backslash looks better.  Make sure to indent the continued line
appropriately.

\subsubsection{One line, one statement}

Don't put more than one statement on the same line:

\begin{quote}
\begin{tabular}{l||l|l}
  &Good                       & Bad \\
  \hline\hline
  1&{\tt if x is True:}       & {\tt if x is True: do\_something()} \\
   &{\tt ~~~~do\_something()} & \\
   && \\
   && \\
  2&{\tt do\_one()}           & {\tt do\_one(); do\_two(); do\_three()} \\
   &{\tt do\_two()}           & \\
   &{\tt do\_three()}         & \\
   && \\
   && \\
  3&{\tt def get\_area(x, y):} & {\tt def get\_area(x, y): return x * y}\\ 
   &{\tt ~~~~return x * y}     & \\
\end{tabular}
\end{quote}


\subsubsection{Blank lines}

Separate classes with three blank lines. Separate methods and functions with
two blank lines. There is also a blank line between the class definition and
the first method definition.

Use blank lines in functions, sparingly, to indicate logical sections.



\subsubsection{Whitespace in expressions and statements}

Surround operators with one white space. Arithmetics operators may be the
exception, for them it is possible to use either one space or none, whatever
reads better. But never use more than one space.

Well, there is another exception, the sign ``='' used in keyword arguments
should not be surrounded by spaces. Type {\tt Document(title="hello")}
instead of {\tt Document(title = "hello")}.

Never add spaces neither before nor after parentheses, brackets or braces.
The only exception is for list comprehensions, where it is allowed to add
a space after the opening bracket, and another space before the closing
bracket.

The comma and colon must be followed by a space (or a new line), but never
put a space before. The only exception is for one element tuples, where
the comma must be immediately followed by the closing parentheses.
The semicolon should never be used.


\section{Comments}

Comments must describe the code that follows them, and must be indented
to the same level of that code. Inline comments are not allowed; this is
to say, a comment always starts a new line.

A comment starts by a single {\tt \#} character followed by a space.

Comments must be written in good english (as good as the developer can
write it). This means, for example, that the first letter must be
capitalized.


\section{Naming conventions}

The names of {\em variables}, {\em classes}, {\em functions}, {\em methods}
and {\em constants} are written with one or more english words. Most of
the words used are {\em nouns}, {\em verbs}, and {\em adjectives}.

Abbreviations may be used, but in general it is preferred the complete
word, for example, {\tt language} instead of {\tt lang}. When an abbreviation
is not obvious, its meaning should be explained with a comment.

The allowed naming conventions are three:

\begin{description}
  \item [lower\_case\_with\_underscores] All words are in lowercase and
    separated by an underscore. This convention is used for {\em variables},
    {\em functions} and {\em methods}.

  \item [UPPER\_CASE\_WITH\_UNDERSCORES] All words are in uppercase and
    separated by an underscore. Used only for {\em constants}.

  \item [CapitalizedWords] All words start by an uppercase, with the
    rest of the word in lowercase. Words are not separated by any
    character, the uppercase letters serve to visually distinguish
    when a new word starts. Used only for {\em classes}.
\end{description}


\subsection{Class names}

Class names are written in capitalized words. Typically they are made of
nouns and/or adjectives.


\subsection{Functions and methods}

Functions and methods are written in lowercase with underscores.

They must start by a verb, and they should be followed by a complenent that
clearifies what the funciont does. For example, it is better to spell
{\tt set\_object} than just {\tt set}.


\subsection{Variables}

Variables are written in lowercase with underscores. Most of the
time they are nouns with or without adjectives.

One letter variables may be used in mathematical expressions, for sequence
indexes, or in comprehensive lists:

\begin{code}
    >>> public_handlers = [x for x in handlers if x.state == 'public']
\end{code}


\subsection{Constants}

Constants are written in uppercase with underscores.
