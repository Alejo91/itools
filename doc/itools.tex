\documentclass[a4paper]{report}

% Import packages
\usepackage[latin1]{inputenc}
\usepackage{color}
\usepackage{psboxit}
\usepackage{verbatim}
\usepackage{itools}

% Document header
\title{\tt itools}
\author{Juan David Ib��ez Palomar\thanks{Permission is granted to copy,
distribute and/or modify this document under the terms of the GNU Free
Documentation License, Version 1.2 or any later version published by the
Free Software Foundation; with no Invariant Sections, no Front-Cover Texts,
and no Back-Cover Texts.  There is a copy of the license at
http://www.gnu.org/copyleft/fdl.html}\\
\small \texttt{jdavid@itaapy.com}}


\begin{document}

\maketitle
\tableofcontents


\chapter{Introduction}

This paper is the official documentation of {\tt itools}.

\section{What is {\tt itools}?}

{\tt itools} is a Python\footnote{\tt http://www.python.org} library, it
groups a number of packages into a single meta-package for easier
development and deployment. The packages included as of {\tt itools 0.10}
are:

\begin{quote}
\begin{tabular}{lll}
  {\tt itools.catalog} & {\tt itools.ical} & {\tt itools.web} \\
  {\tt itools.datatypes} & {\tt itools.resources} & {\tt itools.workflow} \\
  {\tt itools.handlers} & {\tt itools.rss} & {\tt itools.xhtml} \\
  {\tt itools.gettext} & {\tt itools.schemas} & {\tt itools.xliff} \\
  {\tt itools.html} & {\tt itools.tmx} & {\tt itools.xml} \\
  {\tt itools.i18n} & {\tt itools.uri} & \\
\end{tabular}
\end{quote}


\begin{figure}
  \center
  \includegraphics[width=\textwidth]{packages.eps}
  \caption{Dependency diagram}
  \label{Figure: packages}
\end{figure}


It can be seen as an extension of the Python's {\em Standard Library},
as it provides a set of lightly coupled sub-packages which can be used
independently; {\tt itools} is split into the following sub-packages:



\begin{itemize}
  \item {\tt itools.uri} -- an API to manage URIs, to identify and locate
    resources.

  \item {\tt itools.types} -- type marshalers for basic types (integer, date,
    etc.) and not so basic types (filenames, XML qualified names, etc.).

  \item {\tt itools.resources} -- an abstraction layer over resources that
    lets to manage them with a consistent API, independently of where they
    are stored.

  \item {\tt itools.handlers} -- resource handlers infrastructure (resource
    handlers are non persistent classes that add specific semantics to
    resources). This package also includes several handlers {\em out of the
    box}.

  \item {\tt itools.gettext} -- resource handlers for PO and MO files.

  \item {\tt itools.xml} -- includes an intuitive event driven XML parser,
    a handler for XML documents, and the {\bf S}imple {\bf T}emplate
    {\bf L}anguage.

  \item {\tt itools.xhtml} -- resource handlers for XHTML documents.

  \item {\tt itools.html} -- resource handlers for HTML documents.

  \item {\tt itools.i18n} -- tools for language negotiation and text
    segmentation.

  \item {\tt itools.workflow} -- represent workflows as automatons, objects
    can move from one state to another through transitions, classes can add
    specific semantics to states and transitions.

  \item {\tt itools.catalog} -- An index and search engine.
\end{itemize}


\subsection{The resource-handler model}

Of everything in {\tt itools} I am probably most proud of the the three
sub-packages {\tt itools.uri}, {\tt itools.resources} and
{\tt itools.handlers}, which make up what I call the {\em resource-handler}
model.

There is a linear relationship of dependency between these modules, {\tt uri}
depends on nothing but Python, {\tt resources} depends on {\tt uri}, and
{\tt handlers} depends on {\tt resources}.

This allows to use these modules on a flexible way. For example, you can
just use {\tt itools.uri} (ignoring the others) to benefit from a higher
level API to work with URIs than that provided by the Standard Library
({\tt urlparse}).

Or you may use {\tt itools.resources} if you want to benefit from an
abstraction layer over the storage. This is to say, if you want to be able
to manipulate many resources stored in different systems and accessed
through different protocols with the same, consistent, rich API.

You may even take advantage of {\tt itools.handlers} without fully
understanding the model behind. For example, you could use some of the
handlers {\tt itools} offers out-of-the-box for standard file formats
like CSV, PO or XHTML, to simplify your live if you ever need to work
with one of these formats.

But, in order to exploit everything {\tt itools} has to offer to its limit,
you may choose to base part or all of your application architecture on the
{\em resource-handler} model, a model heavily influenced by the filesystem
and the Web. In this situation the three packages ({\tt uri}, {\tt resources}
and {\tt handlers}) show themselves as three different components with
a distinct role in the architecture:

\begin{itemize}
  \item {\tt itools.uri} identifies and locates resources, wherever they
    are. Altogether with {\tt itools.resources} it effectively enables your
    information system to be distributed through an heterogeneous base.

  \item {\tt itools.resources} provides persistency to your data, this is
    to say, resources are where the data is stored.

  \item and {\tt itools.handlers} is where the logic lives. The essential
    characteristic of a resource handler (or handler for short) is that it is
    non-persistent, instead it is associated with a resource, whose content
    it is responsible to manage.
\end{itemize}

The first chapters of this document will cover these sub-packages,
{\tt itools.uri}, {\tt itools.resources} and {\tt itools.handlers} with
detail.

\subsection{eXtensible Markup Language}

The {\tt itools.xml} package depends on {\tt itools.handlers}, this is its
first advantage. It represents an XML document as an DOM like tree. But it
also provides support for schemas, what allows to easily build higher
level data structures.

The package also provides handlers for specific document types out-of-the-box,
most notably XHTML and HTML (even if HTML is not XML, it shares a lot with
XHTML).


\subsection{The Simple Template Language}

The Simple Template Language is included in the {\tt itools.xml} package,
but it is important enough to deserve its own chapter in the documentation.

Its design goals are:

\begin{itemize}
  \item Truly separate logic and presentation. Even Python expressions are
    not allowed within the template.

  \item Really simple. It can be mastered in half a day.

  \item Damn fast (easy to achieve through simplicity).

  \item Secure. Even non-trusted users could write templates without risk,
    because code is not allowed within the template.
\end{itemize}

And the key idea behind, is to make it a {\em descriptive} language.


\subsection{Workflow}

The sub-package {\tt itools.workflow} (once known as {\em flux}) is the
oldest code in {\tt itools}. It does not depend on anything but the
Standard Library, and don't puts any restriction on the storage. This
makes it very easy to combine with other frameworks.

There is a chapter exclusively dedicated to it.

\subsection{Internationalization and Localization}

The {\tt itools.i18n} package provides a wide range of tools for
internationalization and localization of both software and data. From
message extraction to language negotiation, through text segmentation,
fuzzy matching or algorithms to guess the language a text is written in.

A chapter is devoted to this topic.


\subsection{Index and Search}

The {\tt itools.catalog} sub-package provides an index and search engine.
Though still young, it already provides full text indexing, boolean queries
and results sorted by weight.



\section{Who is this document for?}

This document is addressed to Python developers. Though it may be useful
to software architects in general, as some ideas exposed here could be
found interesting\footnote{For example, {\tt itools.workflow} inspired
XXX to write a similar engine in Java, see {\tt http://XXX}}.

It is convenient to have some basic skills with the Python programming
language to fully understand this document. Probably the best introduction
to Python is the official tutorial:

\begin{quote}
  {\tt http://python.org/doc/2.3.5/tut/tut.html}.
\end{quote}

This document touches many different technologies and standards, such as XML.
References will be given in the relevant chapters.


\section{Project status}

Currently {\tt itools} is in alpha stage of development (at the time of
this writing the last version available is {\bf 0.8}). This means that the
API is not yet stable, the documentation is incomplete, and there may be bugs.

However, we have been using it on a daily basis for a long time now. It
fuels the {\em i}Kaaro\footnote{http://www.ikaaro.org} Content Management
System, and most of our customer projects.


\section{Installation}

Before going further, be sure your system is correctly setup. First you
will need Python 2.3.4 or later.

Then download the last {\tt itools} version from {\tt http://www.ikaaro.org},
unpack it somewhere, and install it; {\tt itools} uses {\em distutils}, so
just type:

\begin{code}
    $ python setup.py install
\end{code}

Be sure to have the right permissions.


\section{About this document}

The development keeps ahead of the documentation. I would say the docs cover
around 60\% of {\tt itools}.

There are also two appendixes, one explaining the coding style {\tt itools}
is written in, another one introducing the use of {\em GNU arch}. Both are
specially addressed to those that want to contribute back to the main tree.


\chapter{Uniform Resource Identifiers}

In the wild Internet, the first challenge we encounter is how to identify
and locate the numerous resources that populate it. Well, that's what
{\bf U}niform {\bf R}esource {\bf I}dentifiers (or just URIs) are for.

The Python Standard Library (batteries included!) provides a module named
{\tt urlparse} which is able to split a generic URI into its main parts
(scheme, authority, path, query and fragment), to rebuild it, and to
resolve relative references.

But there are more things we would like to do with a URI. For example we
could go further in the parsing process and split the path into its segments,
then split each segment into the name and the parameters if any; we could
get the user information, host address and port number from the authority;
we could normalize URIs; we could implement other operations beyond just
resolving relative references, etc.

This is the purpose of {\tt itools.uri}, to provide a complete API to parse
and work with URIs, following the standard as described by {\bf RFC2396}.

The main function provided by {\tt itools.uri} is {\tt get\_reference},
a factory to build references:

\begin{api}
    {\tt get\_reference(reference)}\\
    - Parses the given string and returns a reference object.
\end{api}

Let's go right to the code:

\begin{code}
    >>> from itools import uri
    >>> r1 = uri.get_reference('http://www.w3.org/TR/REC-xml/#sec-intro')
    >>> r2 = uri.get_reference('mailto:jdavid@itaapy.com')
    >>> r3 = uri.get_reference('http://www.ietf.org/rfc/rfc2616.txt')
    >>> r4 = uri.get_reference('http://sf.net/cvs/?group_id=5470')
    >>> r5 = uri.get_reference('news:comp.infosystems.www.servers.unix')
\end{code}


\section{Syntax}

Before going further, we will give an overview of the URI syntax, something
required to understand the rest of this chapter.

A URI is divided in two parts. The first one is the scheme: {\tt http},
{\tt ftp}, {\tt mailto}, etc.. The syntax and semantics of the second
part depends on the scheme:

\begin{verbatim}
    uri = <scheme>:<scheme-specific-part>
\end{verbatim}

However, many schemes share a similar syntax for the second part, these
URIs are known as {\em generic URIs}.

\subsection{Generic URIs}

The syntax of a generic URI reference is:

\begin{verbatim}
    <scheme>://<authority><absolute path>?<query>#<fragment>
\end{verbatim}

Generic URIs are modeled by {\tt uri.Reference}. Following the code at the
beginning, we are going to inspect the {\tt r1} object:

\begin{code}
    >>> r1
    <itools.uri.Reference object at 0x403ebc4c>
    >>> print r1
    http://www.w3.org/TR/REC-xml/#sec-intro
    >>> print r1.scheme
    http
    >>> print r1.authority
    www.w3.org
    >>> print r1.path     
    /TR/REC-xml/
    >>> print r1.query

    >>> print r1.fragment
    sec-intro
\end{code}

Note that there is an attribute for every component: the scheme, the
authority, the path, the query and the fragment. Now we are going to
quickly describe each of these components:

\begin{description}
  \item [Scheme] Defines the method or protocol to access the resource.

  \item [Authority] Defines the server address that hosts the resource.
    Its syntax is:

\begin{verbatim}
    authority = [<userinfo>@]<hostport>
\end{verbatim}

  \item [Absolute path] The path identifies the resource within the scope
    of the scheme and authority.

    It consists of a sequence of segments. A segment has two parts, the
    name and the parameters, though the parameters are optional. The syntax is:

\begin{verbatim}
    absolute path = /<relative path>
    relative path = <segment>[/<relative path>]
    segment = <name>[;<parameters>]
\end{verbatim}

  \item [Query] The query is information to be interpreted by the resource.
    It does not have a pre-defined syntax.

  \item [Fragment] The fragment is a reference within the resource.

    Actually, the fragment does not belong to the URI, as it does not help
    to identify the resource, however we include it here because it does
    appears in URI references, what is what we work with.

    As the query, the fragment does not have a pre-defined syntax.
\end{description}


\subsection{Non Generic URIs}

Other schemes do not follow the generic syntax. As an example, let's inspect
the {\tt r2} object seen before:

\begin{code}
    >>> r2 = uri.get_reference('mailto:jdavid@itaapy.com')
    >>> print r2
    mailto:jdavi@itaapy.com
    >>> r2
    <itools.uri.Mailto object at 0x403f45ec>
    >>> print r2.scheme
    mailto
    >>> print r2.username
    jdavid
    >>> print r2.host    
    itaapy.com
\end{code}

As you see the {\tt r2} is not an instance of {\tt uri.Reference}, but an
instance of {\tt uri.Mailto}.


\section{Relative references}

So far the examples we have seen show absolute URIs, but there are relative
URI references too. A relative reference is one that lacks, at least, the
scheme. There are three types of of relative references: network paths,
absolute paths, and relative paths:

\begin{description}
  \item [Network paths] Network paths only lack the scheme, they start by a
    double slash and the authority, followed by the absolute path. They are
    rarely used.

\begin{verbatim}
    //www.ietf.org/rfc/rfc2396.txt
\end{verbatim}

  \item [Absolute paths] The absolute paths lack both the scheme and the
     authority. They start by a slash.

\begin{verbatim}
    /rfc/rfc2396.txt
\end{verbatim}

  \item [Relative paths] Relative paths lack the first slash of absolute
    paths. They can start by the special segment "{\tt .}", or by one or
    more "{\tt ..}". Examples are:

\begin{verbatim}
    rfc/rfc2396.txt
    ./rfc/rfc2396.txt
    ../rfc2616.txt
\end{verbatim}

\end{description}


\subsection{Resolving references}

The most common operation with relative references is to resolve them. That
is to say, to obtain (with the help of a base reference) the absolute
reference that identifies our resource. This is achieved with the
{\tt resolve} method:

\begin{code}
    >>> base = uri.get_reference('http://www.ietf.org/rfc/rfc2615.txt')
    >>> print base.resolve('//www.ietf.org/rfc/rfc2396.txt')
    http://www.ietf.org/rfc/rfc2396.txt
    >>> print base.resolve('/rfc/rfc2396.txt')
    http://www.ietf.org/rfc/rfc2396.txt
    >>> print base.resolve('rfc2396.txt')
    http://www.ietf.org/rfc/rfc2396.txt
\end{code}


\section{Paths}

One component that deserves special attention is the path. The path of a
generic URI is an instance of the {\tt uri.Path} class:

\begin{code}
    >>> ref = uri.get_reference('http://www.ietf.org/rfc/rfc2616.txt')
    >>> ref.path
    <itools.uri.Path at 0x403f50a4>
    >>> print ref.path
    /rfc/rfc2616.txt
\end{code}

Paths are iterable:

\begin{code}
    >>> for segment in ref.path:
    ...     print segment
    ... 
    rfc
    rfc2616.txt
\end{code}

Each component of the path is called a segment. Segments are instances of
the class {\tt uri.Segment}. Each segment has two components, the name and
the parameter. The code below illustrates this:

\begin{code}
    >>> path = uri.Path('/itaapy;lang=es/team')
    >>> for segment in path:
    ...     print repr(segment)
    ...     print '  name:', segment.name
    ...     print '  param:', segment.param
    ... 
    <itools.uri.Segment object at 0x404c1acc>
      name: itaapy
      param: lang=es
    <itools.uri.Segment object at 0x404c1a2c>
      name: team
      param: None
\end{code}

The {\tt uri.Path} class also provides an API to manipulate paths:

\begin{api}
  {\tt is\_absolute()}\\
  - Returns {\tt True} if the path is absolute (i.e. if it starts by an
  slash), {\tt False} otherwise.

  {\tt is\_relative()}\\
  - Returns {\tt True} if the path is relative (i.e. if it does not start
    by a slash), {\tt False} otherwise.

  {\tt get\_prefix(path)}\\
  - Returns the path that is common to {\tt self} and to the given path.

  {\tt resolve(path)}\\
  - Returns a new path from a base path ({\tt self}) and the given path.
  Follows the RFC2396 standard, i.e. takes into account the trailing slash.

  {\tt resolve2(path)}\\
  - Same as {\tt resolve}, but it does not teke into account the trailing
  slash.

  {\tt get\_pathto(path)}\\
  - Returns the path needed to go from {\tt self} to the given path (this
    complements the {\tt resolve} method).

  {\tt get\_pathtoroot()}\\
  - Returns a relative path to the root (something like {\tt ../../..}).
\end{api}


\chapter{Resources}

In the previous chapter we have learned how to work with URIs, objects that
identify resources; but how to manipulate the resources themselves?

The problem we face now is the fact that the resources are dispersed, stored
in different systems, and accessed through different protocols. In spite of
that, it would be nice to be able to manipulate them with a uniform API.

That's the purpose of {\tt itools.resources}, to provide an abstraction layer
over resources: doesn't matters where the resources are stored (in the local
file system, in a remote web server, etc.), they are handled with the same
consistent API.

\section{Retrieving resources}

Of course, the first step, before working with a resource, is to retrieve it,
this is done with the function {\tt get\_resource}:

\begin{api}
  {\tt get\_resource(reference)}\\
  - From the given URI reference returns a resource object.
\end{api}

Let's see a few examples:

\begin{code}
    >>> from itools.resources import get_resource
    >>> get_resource('examples/hello.txt.en')
    <itools.resources.file.File instance at 0x402175ac>
    >>> get_resource('http://example.com')
    <itools.resources.http.File instance at 0x404aadec>
    >>> get_resource('examples')
    <itools.resources.file.Folder instance at 0x401e568c>
\end{code}

These examples show two fundamental aspects of {\tt itools.resources} which
we will explore on detail later. First is the support for multiple protocols,
in particular the example illustrates the schemes {\tt file} for the local
file system, and {\tt http} for the {\bf H}yper{\bf T}ext {\bf T}ransfer
{\bf P}rotocol.

The second aspect is the support for two kinds of resources, files and folders.
Before looking at each type of resource, files and folders, let's see the API
they share:

\begin{api}
  {\tt get\_ctime()}\\
  - Returns a datetime object with the time the resource was created.

  {\tt get\_mtime()}\\
  - Returns a datetime object with the last time the resource was modified.

  {\tt get\_atime()}\\
  - Returns a datetime object with the last time the resource was accessed.

  {\tt get\_mimetype()}\\
  - Returns an string with the mime type of the resource (the mime type of
    folder is always {\tt application/x-not-regular-file}).
\end{api}

It may happen that an specific scheme does not implements one or more of
these methods.

\section{File Resources}

Besides the common methods for files and folders, each resource type has
an specific API, the one for file resources is:

\begin{api}
  {\tt get\_data()}\\
  - Returns the resource data as a byte string.

  {\tt set\_data(data)}\\
  - Replaces the resource content by the given data (a byte string).

  {\tt get\_size()}\\
  - Returns the length (the number of bytes) of the resource data.
\end{api}

As an example of the API let's exercise it with a web page:

\begin{code}
    >>> resource = get_resource('http://example.com')
    >>> print repr(resource)
    <itools.resources.http.File instance at 0x404aadec>
    >>> print resource.get_mimetype()
    text/html
    >>> print resource.get_size()
    438
    >>> print resource.get_data()
    <HTML>
    <HEAD>
      <TITLE>Example Web Page</TITLE>
    </HEAD> 
    <body>  
    <p>You have reached this web page by typing
    ...
\end{code}

Of course, in this example, we can not modify the resource as we don't have
write access to the web server.


\section{Folder Resources}

The specific API for folders is:

\begin{api}
  {\tt get\_resource(path)}\\
  - Returns the resource in the given path (where path is either an instance
    of {\tt uri.Path} or an string).

  {\tt get\_resources(path='.')}\\
  - Returns a list with the names of all the resources in the given path.

  {\tt has\_resource(path)}\\
  - Returns true if there is a resource in the given path, false otherwise.

  {\tt set\_resource(path, resource)}\\
  - Adds the given resource to the given path.

  {\tt del\_resource(path)}\\
  - Removes the resource at the given path.

  {\tt del\_resources(paths)}\\
  - Removes the resources at the given paths (where paths is a list of paths).
\end{api}

Let's exercise the API a little:

\begin{code}
    >>> examples = get_resource('examples')
    >>> print examples.get_resources()
    ['hello.txt.es', 'hello.txt.en', 'CVS']
    >>> hello = examples.get_resource('hello.txt.en')
    >>> print hello.get_data()
    hello world
\end{code}

To copy a web page to the local file system:

\begin{code}
    >>> tmp = get_resource('/tmp')
    >>> web_page = get_resource('http://example.com')
    >>> tmp.set_resource('example.html', web_page)
    >>> copy_of_web_page = tmp.get_resource('example.html')
    >>> print repr(web_page)
    <itools.resources.http.File instance at 0x405e8a2c>
    >>> print repr(copy_of_web_page)
    <itools.resources.file.File instance at 0x405fb64c>
\end{code}

To copy a whole tree:

\begin{code}
    >>> from pprint import pprint
    >>> talks = get_resource('/home/jdavid/talks')
    >>> tmp.set_resource('talks', talks)
    >>> pprint(tmp.get_resources('talks/EuroPython2004'))
    ['itools-vfs',
     'Makefile',
     'itools-vfs.log',
     'itools-vfs.tex',
     'itools-vfs.aux',
     'itools-vfs.toc',
     'itools-vfs.dvi',
     'itools-vfs.ps',
     'itools-vfs.pdf',
     'itools-vfs.tex~']
\end{code}

This is more impressive when copying a big tree from one storage to another.
For example we use it in the context of the {\bf iKaaro} Content Management
System to export and import web sites from the {\bf Z}ope {\bf O}bject
{\bf D}ata{\bf B}ase to the file system, and back from the file system to the
{\bf ZODB}.


\section{Supported schemes}

At the time of this writing the number of supported schemes is pretty short:
{\tt itools.file} for the file system, {\tt itools.http} for the HTTP
protocol (though the implementation is minimal), and {\tt itools.memory}
to store resources in memory. The {\bf iKaaro} CMS provides support to
store resources in Zope 2, and soon in Zope 3.

\subsection{The memory storage}

The memory storage ({\tt itools.memory}) deserves few lines as it is very
handy. It is possible to build a resource from scratch directly calling its
constructor:

\begin{code}
    >>> from itools.resources import memory
    >>> resource = memory.File('hello world')
    >>> resource
    <itools.resources.memory.File instance at 0x405e458c>
    >>> resource.get_data()
    'hello world'
    >>> tmp.set_resource('hello.txt', resource)
\end{code}

This technique is used internally by resource handlers to build handler
instances without passing a resource. They are the object of our next
section.

\chapter{Resource Handlers}

Resources provide persistence to our data, but they lack any knowledge
about the structure and the meaning of the information they contain. For
example, the resource layer ignores that an XML document has a tree
structure, hence it does not provide an API to traverse it; nor it knows
how to get the messages and translations from a PO file.

This is the purpose of the {\em resource handlers}, to add semantics to
specific resources.


\section{Introduction}

Let's start with an example:

\begin{enumerate}
  \item First we load a resource as seen in the previous chapter:
\begin{code}
    >>> from itools.resources import get_resource
    >>> resource = get_resource('http://example.com')
\end{code}

  \item Now we build a handler for the resource:
\begin{code}
    >>> from itools.xml import HTML
    >>> handler = HTML.Document(resource)
    >>> print handler
    <itools.handlers.HTML.Document object at 0x40647b4c>
    >>> print handler.to_str()
    <HTML>
    <HEAD>
      <TITLE>Example Web Page</TITLE>
    </HEAD> 
    <body>  
    <p>You have reached this web page by typing
    ...
\end{code}

  \item Now we can start working with the handler:
\begin{code}
    >>> from itools.xml import XML
    >>> for node in handler.traverse():
    ...     if isinstance(node, XML.Element) and node.name == 'title':
    ...         print node.children
    Example Web Page
\end{code}
\end{enumerate}


\subsection{One resource, many handlers}

The relationship between resources to handlers is {\em 1} to {\em n}.
While you may have several different handlers associated to the same
resource (though this is rarely useful), a handler is only associated
to one resource.

The resource associated to a handler is accessible through the {\tt resource}
attribute:

\begin{code}
    >>> resource
    <itools.resources.http.File instance at 0x404aadec>
    >>> handler.resource
    <itools.resources.http.File instance at 0x404aadec>
    >>> resource is handler.resource
    True
\end{code}


\subsection{The handler skeleton}

As we have seen the handler constructor expects one parameter: the resource
it is meant to handle (and once the handler is built the resource is always
accessible with {\tt handler.resource}).

However, it is also possible to build a handler without passing it any
parameter; in this case a memory resource will be built on the fly, let's
see an example:

\begin{code}
    >>> from itools.xml import HTML
    >>> handler = HTML.Document()
    >>> handler
    <itools.handlers.HTML.Document object at 0x403ee12c>
    >>> handler.resource
    <itools.resources.memory.File instance at 0x40638e4c>
    >>> print handler
    <!DOCTYPE HTML PUBLIC "-//W3C//DTD HTML 4.01 Transitional//EN"
      "http://www.w3.org/TR/html4/loose.dtd">
    <html>
      <head>
        <meta content="text/html; charset=UTF-8" http-equiv="Content-Type">
        <title></title>
      
      <body></body>
    </head></html>
\end{code}

The default content of the resource depends on the handler, and it is called
the {\em handler skeleton}. The constructor also accepts arbitrary keyword
parameters:

\begin{code}
    >>> handler = HTML.Document(title='Hello World')
    >>> print handler
    <!DOCTYPE HTML PUBLIC "-//W3C//DTD HTML 4.01 Transitional//EN">
    <html>
      <head>
        <meta content="text/html; charset=UTF-8" http-equiv="Content-Type"></meta>
        <title>Hello World</title>
      </head>
      <body></body>
    </html>
\end{code}

The keyword parameters are used to initialize the skeleton; in the example
above the {\tt title} parameter defines the HTML document title, which by
default is empty. The parameters accepted depend on the handler.


\section{Overview of the available handlers}

Out of the box {\tt itools} comes with several handlers for different
standard file formats. The Figure~\ref{Figure: handler tree} shows an
excerpt of the tree of the available handler classes, which express the
inheritance relationship between them.

\begin{figure}
  \center
  \includegraphics[height=\textwidth]{handlers.eps}
  \caption{The handler tree}
  \label{Figure: handler tree}
\end{figure}

Follows a short description of these handler classes, including the most
relevant part of the API.

\paragraph{File}

This is the default handler for files, to be used when there is nothing
better. The API is really basic:

\begin{api}
  {\tt to\_str()}\\
  - Returns the resource data as a byte string (this is similar to the
    method {\tt handler.resource.get\_data()}).
\end{api}

\paragraph{Text}

The default handler for text files, whose API is:

\begin{api}
  {\tt to\_unicode(encoding=None)}\\
  - Returns the resource data as an unicode string.

  {\tt to\_str(encoding='UTF-8')}\\
  - Returns the resource data as a byte string, using the given encoding
  (defaults to {\tt UTF-8}). Note that this will be different than the
  string returned by {\tt handler.resource.get\_data()} if the resource
  data is not encoded in {\tt UTF-8} in the source.
\end{api}

\paragraph{XML.Document}

This is the default handler for XML documents, which internally are
represented as a tree. The API includes the methods already described
for the {\tt Text} handler; specific methods of {\tt XML.Document} are:

\begin{api}
  {\tt \_\_cmp\_\_(other)}\\
  - Lets to compare two XML documents.

  {\tt get\_root\_element()}\\
  - Returns the root element of this XML document.

  {\tt traverse()}\\
  - This method allows to traverse the XML document, as it has a tree
    structure. It is a generator which returns a node at a time, starting
    by the document instance.

  {\tt traverse2()}\\
  - As {\tt traverse}, this method allows to traverse the XML document,
    though it is more powerful, and complex. It will be explained in the
    XML chapter.
\end{api}

\paragraph{XHTML.Document}

The handler for XHTML documents, it extends the API provided by
{\tt XML.Document} API with the methods:

\begin{api}
  {\tt get\_head()}\\
  - Returns the head element.

  {\tt get\_body()}\\
  - Returns the body element.

  {\tt to\_text()}\\
  - Strips all the XML markup and returns the text content of the XHTML
    document. This is useful, for example, to index the document.
\end{api}

\paragraph{PO}

The handler to manage PO files, the message catalog of the GNU gettext
utilities.

\begin{api}
  {\tt get\_msgids()}\\
  - Returns the list of message ids stored in the catalog.

  {\tt get\_messages()}\\
  - Returns the list of messages stored in the catalog, where each message
    is represented as an instance of the class {\tt PO.Message}.

  {\tt get\_msgstr(msgid)}\\
  - Returns the message string for the given message id.

  {\tt set\_message(msgid, msgstr=[u''], comments=[u''], references=\{\})}\\
  - Adds a message (from the given parameters) to the catalog.
\end{api}


\section{The handler factory}

So far we have built a handler instance through a handler class:

\begin{code}
    >>> handler = HTML.Document(resource)
    >>> handler
    <itools.xml.HTML.Document object at 0x405740cc>
\end{code}

This is the standard pattern to build instances. However, if the resource
is not an HTML document this procedure would fail. In other words, this
procedure is only useful if you already know the kind of resource you
are working with.

Another option is to let {\tt itools} to choose which handler class to use.
This can be done with the {\tt build\_handler} method, which provides the
factory pattern.

\begin{api}
  {\tt build\_handler(resource)}\\
  - A class method that identifies the given resource, chooses the available
  handler class that better matches it, and builds and returns a handler for
  it.
\end{api}

For example, from the {\tt examples} directory type:

\begin{code}
    >>> from itools.resources import get_resource
    >>> from itools.handlers.Handler import Handler
    >>> from itools import xml
    >>> 
    >>> here = get_resource('.')
    >>> for name in here.get_resource_names():
    ...     resource = here.get_resource(name)
    ...     handler = Handler.build_handler(resource)
    ...     print name, handler
    ... 
    chapter6 <itools.handlers.Folder.Folder object at 0x40538d4c>
    chapter7 <itools.handlers.Folder.Folder object at 0x4032c0cc>
    hello.txt <itools.handlers.Text.Text object at 0x40536f4c>
    hello.xhtml <itools.xml.XHTML.Document object at 0x4053b70c>
    hello.xml <itools.xml.XML.Document object at 0x40538d6c>
\end{code}

Note that {\tt build\_handler} is a class method. In the example above
we have called it through the most abstract handler class: {\tt Handler},
which is the root of the inheritance tree. But it is also possible to call
it with another handler class:

\begin{code}
    >>> from itools.xml import XML
    >>> 
    >>> resource = here.get_resource('hello.xhtml')
    >>> XML.Document.build_handler(resource)
    <itools.xml.XHTML.Document object at 0x405c6ecc>
\end{code}

There is an important difference between calling {\tt build\_handler} from
one or another class: the set of possible handler classes to use is
restricted to all the sub-classes of the choosen handler class. For example,
if we pass a plain text file to {\tt XML.Document.build\_handler}, it will
fail:

\begin{code}
    >>> resource = here.get_resource('hello.txt')
    >>> XML.Document.build_handler(resource)
    Traceback (most recent call last):
      [...]
    xml.parsers.expat.ExpatError: syntax error: line 1, column 0
\end{code}


\subsection{The {\tt get\_handler} shorthand}

There is a short way to load a handler (instead of loading first the resource
and then building the handler explicitly), the function {\tt get\_handler}:

\begin{api}
  {\tt get\_handler(uri)}\\
  - Loads the resource at the given uri, tries to guess its mimetype by
  different meanings (name extension, etc.), searches for a suitable
  handler class in the registry, builds and returns the handler for
  the resource.
\end{api}

Compare the explicit way seen before:

\begin{code}
    >>> from itools.resources import get_resource
    >>> from itools.handlers.Handler import Handler
    >>>
    >>> resource = get_resource('http://example.com')
    >>> handler = Handler.build_handler(resource)
\end{code}

With the shorthand:

\begin{code}
    >>> from itools.handlers import get_handler
    >>>
    >>> handler = get_handler('http://example.com')
\end{code}


\section{Folders}

A handler that deserves its particular section is the default handler for
folders, whose core API is:

\begin{api}
  {\tt get\_handler(path)}\\
  - Returns a handler for the resource at the given path. It will use the
    available handler class that better matches the resource mimetype.

  {\tt get\_handler\_names(path='.')}\\
  - Returns a list with the names of all the handlers in the given path.

  {\tt get\_handlers(path='.')}\\
  - Returns the handlers in the given path (it is a generator).

  {\tt has\_handler(path)}\\
  - Returns {\tt True} if there is a handler in the given path, {\tt False}
    otherwise.

  {\tt set\_handler(path, handler)}\\
  - Adds the given handler to the given path. Actually what is added is
    the resource associated to the handler.

  {\tt del\_handler(path)}\\
  - Removes the handler at the given path (i.e. the associated resource).
\end{api}

An example will show it better.

\begin{enumerate}
  \item First we build a handler for the temporary directory:

\begin{code}
    >>> from itools.handlers import get_handler
    >>>
    >>> tmp = get_handler('/tmp')
    >>> tmp
    <itools.handlers.Folder.Folder object at 0x40652dec>
    >>> tmp.resource
    <itools.resources.file.Folder instance at 0x406acacc>
\end{code}

  \item Second, we create a new HTML handler:

\begin{code}
    >>> from itools.xml import HTML
    >>> hello = HTML.Document(title='Hello World')
    >>> hello.resource
    <itools.resources.memory.File instance at 0x405e49cc>
\end{code}

    Note that the associated resource is built on the fly and lives in memory.

  \item Third, we set the HTML handler to the temporary folder:

\begin{code}
    >>> tmp.set_handler('hello.html', hello)
\end{code}

    What this actually does is to add the file {\tt hello.resource} (which
    lives in memory) to the folder {\tt tmp.resource} (which is on the
    file system); this is to say, it creates a new file in the file system
    at {\tt /tmp/hello.html}.

  \item Finally, we get the handler we just added:

\begin{code}
    >>> hello = tmp.get_handler('hello.html')
    >>> hello.resource
    <itools.resources.file.File instance at 0x40638c6c>
\end{code}

    Note that the handler {\tt hello} we have built in these last lines
    manages a resource that lives in the file system.

\end{enumerate}



\subsection{The handler tree}

Folders allow to classify files, hence giving a tree structure to our data.
Every handler has two attributes, {\tt parent} and {\tt name}, they tell us
where the handler is in the handler tree:

\begin{code}
    >>> hello.parent
    <itools.handlers.Folder.Folder object at 0x403ebb2c>
    >>> hello.name  
    'hello.html'
    >>> hello.parent is tmp
    True
    >>> print tmp.parent
    None
    >>> print tmp.name

    >>> 
\end{code}

Based on these two attributes handlers provide the following API:

\begin{api}
  {\tt get\_abspath()}\\
  - Returns the absolute path from the tree root to the {\tt self} handler.

  {\tt get\_root()}\\
  - Returns the handler for the root of the tree.

  {\tt get\_pathtoroot()}\\
  - Returns a relative path from {\tt self} to the tree root (e.g.
    {\tt ../../..}).

  {\tt get\_pathto(handler)}\\
  - Returns a relative path from {\tt self} to the given handler (which is
    supposed to be in the same tree), for example: {\tt ../../zoo/lion}.

  {\tt traverse()}\\
  - This method allows to traverse the handler tree below this folder. It
    is a generator which returns a handler at a time, starting by this
    folder.

  {\tt acquire(name)}\\
  - If the current handler is a folder and contains a resource with the given
    name, then return a handler for it; otherwise look at the parent folder,
    and recursively to the root tree. This method actually shows how to
    implement {\em acquisition}.
\end{api}

\chapter{eXtensible Markup Language}

The purpose of this chapter is to explain the XML services provided by
{\tt itools}, which can be found in the sub-package {\tt itools.xml}.

\section{XML.Document}

The XML facilities provided by {\tt itools} are built upon the resource-handler
architecture. The basic handler class that enables us to manipulate XML files
is {\tt XML.Document}, now we are going to play a little bit with it:

\begin{code}
    >>> from itools.handlers import get_handler
    >>> import itools.xml
    >>>
    >>> doc = get_handler('examples/hello.xml')
    >>> doc
    <itools.xml.XML.Document object at 0x4064466c>
    >>> print doc.xml_version
    1.0
    >>> print doc.standalone 
    -1
    >>> print doc.document_type
    None
    >>> print doc.root_element
    <itools.xml.XML.Element object at 0xb7a3272c>
\end{code}

The code above shows the four attributes that keep the document's state:

\begin{api}
    {\tt xml\_version}\\
    - The XML version of the document, usually it is {\tt 1.0}.

    {\tt standalone}\\
    - Possible values are: {\tt 1} if the document was declared standalone,
      {\tt 0} if it was declared not to be standalone, or {\tt -1} if the
      standalone clause was omitted.

    {\tt document\_type}\\
    - If the document lacks a document type declaration this attribute will
      be {\tt None}. If the document type was specified this attribute will
      be a tuple with four values: the name, the system id, the public id
      and a boolean that tells wether the document contains an internal
      declaration subset.

    {\tt root\_element}\\
    - An instance of the {\tt XML.Element} class, the root of the DOM-like
      tree that represents the XML data, and that we will study later with
      more detail.
\end{api}


The API for the documents is rather simple:

\begin{api}
    {\tt get\_root\_element()}\\
    - Returns the root element.

    {\tt traverse()}\\
    - A generator that traverses the XML tree in pre-order, and returns
      each time a node. It is a shorthand for {\tt root\_element.traverse()}.

    {\tt traverse2()}\\
    - A more powerful version of {\tt traverse}. It is a shorthand for
      {\tt root\_element.traverse2()}.
\end{api}


\subsection{Inspecting the tree}

Coming back to the example, the {\tt examples/hello.xml} file's content is:

\begin{code}
    <?xml version="1.0" encoding="UTF-8"?>
    <html>
      <head>
        <meta http-equiv="Content-Type" content="text/html; charset=UTF-8" />
        <title>Hello world</title>
        <!-- Changed by: , 02-Jun-2004 -->
      </head>
      <body>
      </body>
    </html>
\end{code}

The method {\tt traverse} lets us to easily inspect the tree nodes:

\begin{code}
    >>> for node in doc.traverse():
    ...     print repr(node)
    ... 
    <itools.xml.XML.Element object at 0xb7a3316c>
    u'\n  '
    <itools.xml.XML.Element object at 0xb7a331ac>
    u'\n    '
    <itools.xml.XML.Element object at 0xb7a332ac>
    u'\n    '
    <itools.xml.XML.Element object at 0xb7a331ec>
    u'Hello world'
    u'\n    '
    <itools.xml.XML.Comment object at 0xb7a335cc>
    u'\n  '
    u'\n  '
    <itools.xml.XML.Element object at 0xb798574c>
    u'\n  '
    u'\n'
\end{code}

Here we see the three kind of nodes currently supported: elements, comments
and text nodes.

\subsection{Elements}

Of the three kinds of nodes, elements are the most most important and complex.
Element nodes give the tree structure, as they are the only ones that may have
children. Following the example, lets look inside an element:

\begin{code}
    >>> root_element = doc.root_element
    >>> root_element
    <itools.xml.XML.Element object at 0xb7a3272c>
\end{code}



Lets to stop and look at the API of the most important node type,
{\tt XML.Element}:

\begin{api}
    {\tt parent}\\
    - It is the element's parent.

    {\tt attributes}\\
    - The element's attributes, it is a mapping from attribute names to
    {\tt XML.Attribute} objects.

    {\tt get\_elements(name=None)}\\
    - Returns a list with all the element nodes whose name is the given
    name; if the parameter {\tt name} is not given, then return all the
    element nodes.

    {\tt traverse()}\\
    - As with {\tt XML.Document}, this method is a generator that traverses
    the element children in pre-order, and returns each time a node.

    {\tt copy()}\\
    - Returns a copy of the element, including all its children; but with
    the attribute {\tt parent} set to {\tt None}.

    {\tt append\_child(node)}\\
    - Appends the given node to the element children, the current element will
    be the node's parent.
\end{api}


\section{Namespaces}

One very important concept of XML are the {\em namespaces}.

An XML document may contain many different elements and attributes, if there
is not any XML declaration these elements and attributes will be instances
of the {\tt XML.Element} and {\tt XML.Attribute} classes, with the standard
API and none specific semantics.

But if there are namespace declarations, {\tt itools.xml} will be able to
build specific objects that add some special semantics. To see it we are
going to load the file {\tt examples/hello.xhtml}, whose content is:

\begin{code}
    <?xml version="1.0" encoding="UTF-8"?>
    <!DOCTYPE html PUBLIC "-//W3C//DTD XHTML 1.0 Strict//EN"
           "http://www.w3.org/TR/xhtml1/DTD/xhtml1-strict.dtd">
    <html xmlns="http://www.w3.org/1999/xhtml">
      <head>
        <meta http-equiv="Content-Type" content="text/html; charset=UTF-8" />
        <title>Hello world</title>
        <!-- Changed by: , 02-Jun-2004 -->
      </head>
      <body>
      </body>
    </html>
\end{code}

The only two differences between this file and the example we saw at the
beginning ({\tt examples/hello.xml}) are the {\em document type} declaration
and, what interests us now, the XML namespace declaration:

\begin{code}
    xmlns="http://www.w3.org/1999/xhtml"
\end{code}

What this sentence says is that all the elements and attributes within the
document belong to the XHTML namespace. This information lets the {\tt itools}
XML parser to build a more ``intelligent'' tree:

\begin{code}
    >>> resource = get_resource('examples/hello.xhtml')
    >>> doc = XML.Document(resource)
    >>> 
    >>> doc
    <itools.xml.XML.Document object at 0x405ea38c>
    >>> 
    >>> for node in doc.traverse():
    ...     print repr(node)
    ... 
    <itools.xml.XHTML.Element object at 0xb7a3316c>
    u'\n  '
    <itools.xml.XHTML.Element object at 0xb7a331ac>
    u'\n    '
    <itools.xml.XHTML.Element object at 0xb7a332ac>
    u'\n    '
    <itools.xml.XHTML.Element object at 0xb7a33f2c>
    u'Hello world'
    u'\n    '
    <itools.xml.XML.Comment object at 0xb798554c>
    u'\n  '
    u'\n  '
    <itools.xml.XHTML.Element object at 0xb79855ac>
    u'\n  '
    u'\n'
\end{code}

Now, the element nodes are not any more instances of {\tt XML.Element}, but
instances of {\tt XHTML.Element}, which extends the generic API with the
methods:

\begin{api}
    {\tt is\_inline()}\\
    - Returns {\tt True} if the element is an inline element, {\tt False}
    otherwise.

    {\tt is\_block()}\\
    - Returns {\tt True} if the element is a block element, {\tt False}
    otherwise.
\end{api}

Ok, not too much\footnote{These two methods, {\tt is\_inline} and
{\tt is\_block}, are actually really useful. They are used by the message
extraction algorithm, a fundamental brick of the internationalization and
localization services provided by {\tt itools}.}, but enough to give an
idea of the power of {\tt itools.xml}. In the next chapter we will see a
much more compelling example of what can be done with {\tt itools.xml},
the {\bf S}imple {\bf T}emplate {\bf L}anguage.



\chapter{Simple Template Language}

{\bf STL} is a template language. It is implemented as an XML namespace
handler, taking avantage of the underlying infrastrucutre provided by
{\tt itools.xml}.

{\bf STL} process and transforms XML files. It is aimed at presentation,
for example to produce the web pages that make up the user interface of
a web application.


Unlike other template languages in the Python world, {\bf STL} does not
mix Python code within the template. The {\bf STL} statements only describe
the transformations to be performed on the template. It is a {\em descriptive}
language.

For example, look at the template below (see {\tt example/template.xml}):

\begin{code}
    <?xml version="1.0" encoding="UTF-8"?>
    <!DOCTYPE html
         PUBLIC "-//W3C//DTD XHTML 1.0 Transitional//EN"
         "http://www.w3.org/TR/xhtml1/DTD/xhtml1-transitional.dtd">
    <html xmlns="http://www.w3.org/1999/xhtml"
          xmlns:stl="http://xml.itools.org/namespaces/stl">
      <head></head>
      <body>
        <h1 stl:content="title" />
      </body>
    </html>
\end{code}

Note the declaration of the {\em stl} namespace.

When this template will be processed, the content of the {\tt <h1>} tag
will be replaced by the value of the variable {\tt title}. But, which
is its value?

\section{The namespace}

In order to process an {\bf STL} template, you need to pass it a Python
namespace. But first we have to load the template as a handler:

\begin{code}
    >>> from itools.resources import get_resource
    >>> from itools.xml import XML
    >>>   
    >>> resource = get_resource('examples/template.xml')
    >>> template = XML.Document(resource)
    >>> 
    >>> template
    <itools.xml.XML.Document object at 0x405ea38c>
\end{code}

Note that so far nothing delates the {\bf STL} presence, but inspecting
the {\tt template} object shows that the {\em stl} namespace handler has
been loaded:

\begin{code}
    >>> template.stl
    <itools.xml.STL.STL object at 0x406516cc>
\end{code}

Ok, it is time to build the namespace and process the template:

\begin{code}
    >>> namespace = {'title': 'hello world'}
    >>> print template.stl(namespace)
    <?xml version="1.0" encoding="UTF-8"?>
    <!DOCTYPE html
         PUBLIC "-//W3C//DTD XHTML 1.0 Transitional//EN"
        "http://www.w3.org/TR/xhtml1/DTD/xhtml1-transitional.dtd">
    <html xmlns:stl="http://xml.itools.org/namespaces/stl"
          xmlns="http://www.w3.org/1999/xhtml">
      <head></head>
      <body>
        <h1>hello world</h1>
      </body>
    </html>
\end{code}


As this example shows, the value of the variable {\tt title} is looked within
the namespace passed as parameter.


\section{Repeat}

Let's see a more complex example. The template below shows a simple
adressbook (see {\tt examples/addressbook.xml}):

\begin{code}
    <?xml version="1.0" encoding="UTF-8"?>
    <!DOCTYPE html
         PUBLIC "-//W3C//DTD XHTML 1.0 Transitional//EN"
         "http://www.w3.org/TR/xhtml1/DTD/xhtml1-transitional.dtd">
    <html xmlns="http://www.w3.org/1999/xhtml"
          xmlns:stl="http://xml.itools.org/namespaces/stl">
      <head></head>
      <body>
        <h3>Addressbook</h3>
        <ul>
          <li stl:repeat="address addressbook">
            <stl:block content="address/last_name" />,
            <stl:block content="address/first_name" />:
            <stl:block content="address/telephone" />
          </li>
        </ul>
      </body>
    </html>
\end{code}

The first new thing this example shows is the {\tt repeat} statement. While
{\tt stl:content} expects an string as the value, {\tt stl:repeat} expects
a sequence. When this template is processed, the XML output will contain as
many {\tt <li>} elements as items are in the {\tt addressbook} variable.
Within the element, in each iteration over the {\tt addressbook} sequence,
the variable {\tt address} will be the respective item of the list.

The second new thing we see is the {\tt <stl:block>} element. When the
template is processed the {\tt <stl:block>} tags are automatically
removed.

Finally, look at the expression {\tt address/last\_name}, it shows the {\em
slash} operator, which lets to traverse namespaces. So the variable {\tt
address} is expected to be a dictionary, and {\tt lastname} a key in that
mapping, whose value is an string.

\subsection{The Python side}

Now let's see the Python code (in the script {\tt examples/addressbook.py}),
this time as a class which keeps the addressbook information, with a method
that process the template to produce an HTML page that shows the addressbook.

\begin{code}
    # Import from itools
    from itools.resources import get_resource
    from itools.xml import XML


    class Addressbook(object):
        def __init__(self):
            self.addresses = []


        def add_address(self, last_name, first_name, telephone):
            address = {'last_name': last_name,
                       'first_name': first_name,
                       'telephone': telephone}
            self.addresses.append(address)


        def view(self):
            # Load the STL template
            resource = get_resource('addressbook.xml')
            template = XML.Document(resource)

            # Build the namespace
            namespace = {'addressbook': self.addresses}

            # Process the template and return the output
            return template.stl(namespace)
\end{code}

The end of the module creates an {\tt Addressbook} instance, fills it with
a couple of entries and prints the result of the {\tt view} method:

\begin{code}
    if __name__ == '__main__':
        # Create the addressbook
        addressbook = Addressbook()
        addressbook.add_address('Jordan', 'Robert', '0606060606')
        addressbook.add_address('Buendia', 'Aureliano', '0612345678')

        # Output the addressbook content
        print addressbook.view()
\end{code}

The output of running this program is:

\begin{code}
    <?xml version="1.0" encoding="UTF-8"?>
    <!DOCTYPE html
         PUBLIC "-//W3C//DTD XHTML 1.0 Transitional//EN"
        "http://www.w3.org/TR/xhtml1/DTD/xhtml1-transitional.dtd">
    <html xmlns:stl="http://xml.itools.org/namespaces/stl"
          xmlns="http://www.w3.org/1999/xhtml">
      <head></head>
      <body>
        <h3>Addressbook</h3>
        <ul>
          <li>
            Jordan,
            Robert:
            0606060606
          </li><li>
            Buendia,
            Aureliano:
            0612345678
          </li>
        </ul>
      </body>
    </html>
\end{code}









\chapter{Conclusion}

This article has presented three core features provided by {\tt itools}:
{\tt itools.uri}, {\tt itools.resources} and {\tt itools.handlers}. Many
things have been left aside to keep the article short: the description
of the possibilities of the XML handler, or a deeper discussion about how
to build your own handlers.

We hope you have found these tools interesting enough to get a closer look.

\section{In the real world}

In spite of the youth of {\tt itools} (version {\tt 0.3} at the time of
this writing), it is already being used in real production projects.
Specially, it is the foundation of the {\bf iKaaro} Content Management System.

Some of the advantages provided by its architecture are:

\begin{itemize}
  \item The whole web site data is stored as a tree of human readable
    files, instead of opaque persistent Python objects. It is possible
    to export a web site to the file system, to inspect and manipulate
    the tree, to make a tarball, to import it back to the ZODB.

  \item This approach encourages the use of standards, every time we
    need a new feature we look first if there is already an standard
    file format. The result is a highly portable, standards compliant
    application, easy to interact with other applications.

  \item It is possible to seamlessly integrate content from foreign
    sources and to make it appear as if they were in our web site.
    We are just at the surface of the possibilities.
\end{itemize}

But the most important, it provides a simple, high-productive framework
to build our applications on, to reduce the development time to the
minimum.

\section{Future works}

There is much to do however:

\begin{itemize}
  \item {\tt itools.uri} is not yet 100\% compliant with the RFC2396,
    once mature enough this module even could become part of the
    Python's Standard Library;

  \item support for new schemes, protocols and storages should be
    added to {\tt itools.resources};

  \item there are a thousand handlers to develop, for standard file formats
    like vCard, iCal, etc. 
\end{itemize}

And, last but not least, {\tt itools} is free software, released to the
world under the terms and conditions of the GNU Lesser General Public
License, developers own the copyright of the code they write, contributions
are very welcomed.


\begin{appendix}
\chapter{GNU arch}

The {\tt itools} source code is managed by {\em tla} (also known as
{\em GNU arch}\footnote{\tt http://www.gnu.org/software/gnu-arch/}),
a distributed control version system.

This chapter explains how:

\begin{enumerate}
  \item To keep track of the {\tt itools} development.

  \item To maintain private changes.

  \item To contribute your changes back to the main development tree.
\end{enumerate}


\section{Keeping track of {\tt itools}}

You may want to have the last bleeding edge features from {\tt itools} in
your system as soon as they are written, or to track how the development
is going on. Then this section is for you.

\subsection{Browsing the sources}

To browse the {\tt itools} archive tree through the web, just go the url
below:

\begin{code}
    http://in-girum.net/cgi-bin/archzoom.cgi/jdavid@itaapy.com--public
\end{code}

\subsection{Check out}

To check out {\tt itools} from the archive you need to install {\em tla}.
Most distributions include it, for example, if you use
Gentoo\footnote{http://www.gentoo.org} just type:

\begin{code}
    $ sudo emerge tla
\end{code}

Once {\em tla} is installed, follow the steps described below.


\subsubsection{Set your id}

\begin{code}
    $ tla my-id "Toto Bonaparte <toto@example.com>"
    $ tla my-id
    Toto Bonaparte <toto@example.com>
\end{code}


\subsubsection{Register the official {\tt itools} archive}

\begin{code}
    $ tla register-archive jdavid@itaapy.com--public \
          http://in-girum.net/~jdavid/archives/public
    $ tla archives
    jdavid@itaapy.com--public
        http://in-girum.net/~jdavid/archives/public
\end{code}


\subsubsection{Check out {\tt itools}}

\begin{code}
    $ cd ~/sandboxes
    $ tla get jdavid@itaapy.com--public/itools--main--0.4 itools-0.4
    $ cd itools-0.4
    $ tla tree-version
    jdavid@itaapy.com--public/itools--main--0.4
\end{code}

\subsubsection{A session with {\em tla} and {\tt itools}}

Now, whenever you want to see if something has changed in {\tt itools},
just type:

\begin{code}
    $ cd ~/sandboxes/itools-0.4
    $ tla missing --summary
    patch-80
        use Python's documentation to profile the catalog
    patch-81
        fix XML error handling (hence better STL message errors)
\end{code}

The output shows the new patches available (if your code is up-to-date
the output will be empty). Say you want to apply the patches, type:

\begin{code}
    $ tla update
    [...]
\end{code}

Use the {\tt tla help} to learn about other commands available.


\section{Maintaining private changes}

Now maybe you want to make some changes to {\tt itools}. The wisest to do
in this situation is to create a branch of {\tt itools}, this will let you
to easily update to the last version while keeping your changes.

The first step is to setup an archive (if you have already one you can
skip to the next subsection).

\subsubsection{Create an archive}

\begin{code}
    $ mkdir ~/{archives}
    $ mkdir ~/{archives}/public
    $ tla make-archive toto@example.com--public ~/{archives}/public
    $ tla archives
    jdavid@itaapy.com--public
        http://in-girum.net/~jdavid/archives/public
    toto@example.com--public
        /home/toto/{archives}/public
\end{code}

Make it your default archive:

\begin{code}
    $ tla my-default-archive toto@example.com--public
    $ tla my-default-archive
    toto@example.com--public
\end{code}

\subsubsection{Create a branch}

With your own archive, it is time to create a branch of {\tt itools}:

\begin{code}
    $ tla archive-setup itools--toto--0.4
      * creating category toto@example.com--public/itools
      * creating branch toto@example.com--public/itools--toto
      * creating version toto@example.com--public/itools--toto--0.4
    $ tla tag jdavid@itaapy.com--public/itools--main--0.4 itools--toto--0.4
      * Archive caching revision
\end{code}

So now you can replace the check-out from the main tree with a one from
your own branch:

\begin{code}
    $ cd ~/sandboxes
    $ rm -rf itools-0.4
    $ tla get toto@example.com--public/itools--toto--0.4 itools-0.4
    $ cd itools-0.4
    $ tla tree-version
    toto@example.com--public/itools--toto--0.4
\end{code}


\subsubsection{Working with your branch}

So, now you modify {\tt itools} to add a new feature. Before anything else
it is a good idea to check what you have changed:

\begin{code}
    $ tla changes
    [...]
\end{code}

This command shows which files (and folders) have been modified, removed,
added or moved. For a more detailed description, try:

\begin{code}
    $ tla changes --diffs | less
\end{code}

Before committing, it is also a very good idea to check wether you forgot
to add a file:

\begin{code}
    $ tla tree-lint
    [...]
\end{code}

The command above checks your project tree, it will tell you about files
suspected to be source code.

When you are sure everything is alright, it came the time to commit.
First you have to write a log message:

\begin{code}
    $ vi `tla make-log`
\end{code}

Within the editor, you should introduce a title that describes the changes
you have done, and optionally a longer description, and some keywords. Once
you are done, left the editor and type:

\begin{code}
    $ tla commit
    $ tla revisions
    [...]
    patch-1
        add feature XXX
\end{code}


\subsubsection{Merging from the main branch}


Ok, so now the upstream version of {\tt itools} is modified, how to merge
the changes in your tree? easy:

\begin{code}
    $ cd ~/sandboxes/itools-0.4
    $ tla star-merge -t jdavid@itaapy.com--public/itools--main--0.4
    [...]
\end{code}

Beware, there may be conflicts that you must resolve.

Now, your project tree contains the changes from the upstream archive,
you must commit them in your own archive. The log is written automatically
by typing:

\begin{code}
    $ tla log-for-merge >> `tla make-log`
    $ vi `tla make-log`
\end{code}

Within the editor there will be a description detailing the patchs that
have been applied. So you just have to add the subject, something like
``merging from the main tree''. Once this is done just commit as usual:

\begin{code}
    $ tla commit
\end{code}


\section{Contibuting your work to the main tree}
\end{appendix}



\end{document}