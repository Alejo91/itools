\chapter{GNU arch}

The {\tt itools} source code is managed by {\em tla} (also known as
{\em GNU arch}\footnote{\tt http://www.gnu.org/software/gnu-arch/}),
a distributed control version system.

This chapter explains how:

\begin{enumerate}
  \item To keep track of the {\tt itools} development.

  \item To maintain private changes.

  \item To contribute your changes back to the main development tree.
\end{enumerate}


\section{Keeping track of {\tt itools}}

You may want to have the last bleeding edge features from {\tt itools} in
your system as soon as they are written, or to track how the development
is going on. Then this section is for you.

\subsection{Browsing the sources}

To browse the {\tt itools} archive tree through the web, just go the url
below:

\begin{code}
    http://in-girum.net/cgi-bin/archzoom.cgi/jdavid@itaapy.com--public
\end{code}

\subsection{Check out}

To check out {\tt itools} from the archive you need to install {\em tla}.
Most distributions include it, for example, if you use
Gentoo\footnote{http://www.gentoo.org} just type:

\begin{code}
    $ sudo emerge tla
\end{code}

Once {\em tla} is installed, follow the steps described below.


\subsubsection{Set your id}

\begin{code}
    $ tla my-id "Toto Bonaparte <toto@example.com>"
    $ tla my-id
    Toto Bonaparte <toto@example.com>
\end{code}


\subsubsection{Register the official {\tt itools} archive}

\begin{code}
    $ tla register-archive jdavid@itaapy.com--public \
          http://in-girum.net/~jdavid/archives/public
    $ tla archives
    jdavid@itaapy.com--public
        http://in-girum.net/~jdavid/archives/public
\end{code}


\subsubsection{Check out {\tt itools}}

\begin{code}
    $ cd ~/sandboxes
    $ tla get jdavid@itaapy.com--public/itools--main--0.4 itools-0.4
    $ cd itools-0.4
    $ tla tree-version
    jdavid@itaapy.com--public/itools--main--0.4
\end{code}

\subsubsection{A session with {\em tla} and {\tt itools}}

Now, whenever you want to see if something has changed in {\tt itools},
just type:

\begin{code}
    $ cd ~/sandboxes/itools-0.4
    $ tla missing --summary
    patch-80
        use Python's documentation to profile the catalog
    patch-81
        fix XML error handling (hence better STL message errors)
\end{code}

The output shows the new patches available (if your code is up-to-date
the output will be empty). Say you want to apply the patches, type:

\begin{code}
    $ tla update
    [...]
\end{code}

Use the {\tt tla help} to learn about other commands available.


\section{Maintaining private changes}

Now maybe you want to make some changes to {\tt itools}. The wisest to do
in this situation is to create a branch of {\tt itools}, this will let you
to easily update to the last version while keeping your changes.

The first step is to setup an archive (if you have already one you can
skip to the next subsection).

\subsubsection{Create an archive}

\begin{code}
    $ mkdir ~/{archives}
    $ mkdir ~/{archives}/public
    $ tla make-archive toto@example.com--public ~/{archives}/public
    $ tla archives
    jdavid@itaapy.com--public
        http://in-girum.net/~jdavid/archives/public
    toto@example.com--public
        /home/toto/{archives}/public
\end{code}

Make it your default archive:

\begin{code}
    $ tla my-default-archive toto@example.com--public
    $ tla my-default-archive
    toto@example.com--public
\end{code}

\subsubsection{Create a branch}

With your own archive, it is time to create a branch of {\tt itools}:

\begin{code}
    $ tla archive-setup itools--toto--0.4
      * creating category toto@example.com--public/itools
      * creating branch toto@example.com--public/itools--toto
      * creating version toto@example.com--public/itools--toto--0.4
    $ tla tag jdavid@itaapy.com--public/itools--main--0.4 itools--toto--0.4
      * Archive caching revision
\end{code}

So now you can replace the check-out from the main tree with a one from
your own branch:

\begin{code}
    $ cd ~/sandboxes
    $ rm -rf itools-0.4
    $ tla get toto@example.com--public/itools--toto--0.4 itools-0.4
    $ cd itools-0.4
    $ tla tree-version
    toto@example.com--public/itools--toto--0.4
\end{code}


\subsubsection{Working with your branch}

So, now you modify {\tt itools} to add a new feature. Before anything else
it is a good idea to check what you have changed:

\begin{code}
    $ tla changes
    [...]
\end{code}

This command shows which files (and folders) have been modified, removed,
added or moved. For a more detailed description, try:

\begin{code}
    $ tla changes --diffs | less
\end{code}

Before committing, it is also a very good idea to check wether you forgot
to add a file:

\begin{code}
    $ tla tree-lint
    [...]
\end{code}

The command above checks your project tree, it will tell you about files
suspected to be source code.

When you are sure everything is alright, it came the time to commit.
First you have to write a log message:

\begin{code}
    $ vi `tla make-log`
\end{code}

Within the editor, you should introduce a title that describes the changes
you have done, and optionally a longer description, and some keywords. Once
you are done, left the editor and type:

\begin{code}
    $ tla commit
    $ tla revisions
    [...]
    patch-1
        add feature XXX
\end{code}


\subsubsection{Merging from the main branch}


Ok, so now the upstream version of {\tt itools} is modified, how to merge
the changes in your tree? easy:

\begin{code}
    $ cd ~/sandboxes/itools-0.4
    $ tla star-merge -t jdavid@itaapy.com--public/itools--main--0.4
    [...]
\end{code}

Beware, there may be conflicts that you must resolve.

Now, your project tree contains the changes from the upstream archive,
you must commit them in your own archive. The log is written automatically
by typing:

\begin{code}
    $ tla log-for-merge >> `tla make-log`
    $ vi `tla make-log`
\end{code}

Within the editor there will be a description detailing the patchs that
have been applied. So you just have to add the subject, something like
``merging from the main tree''. Once this is done just commit as usual:

\begin{code}
    $ tla commit
\end{code}


\section{Contibuting your work to the main tree}

To contribute your changes back to the main development branch you must
make your branch available through internet. We assume the archive you
have set-up is your local computer, so you have to create a mirror of
your archive from your local computer to an internet server:

\begin{code}
    $ tla make-archive --listing --mirror toto@example.com--public \
          sftp://example.com/home/toto/{archives}/public
\end{code}

You must be sure your mirror can be accessed through an internet protocol,
like HTTP or FTP. So other developers will be able to merge from your branch.

Now, whenever you make and commit a change, to share that change with the
community, you have to synchronize your mirror:

\begin{code}
    $ tla archive-mirror
    [...]
\end{code}

