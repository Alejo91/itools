\chapter{The Web}

This chapter documents the high-level, cross-protocol, programming interface
provided by {\tt itools.web} to develop web applications.


\section{The Publisher}

The first and most important interface within the Web is the URI (Uniform
Resource Identifier), both from the user's and programmer's point of view.
This is how a URI looks like with {\tt itools.web}:

\begin{code}
    http://localhost:8000/users/toto/;view
\end{code}

What is of most interest to us is the path, which expressed in a general
form has the structure:

\begin{code}
    <path to resource>/;<method>
\end{code}

It is splitted in two parts, a path to a resource, and an action or view
over that resource. The Figure~\ref{Figure: web tree} shows an excerpt
of the application's structure, the excerpt that is relevant to our example.

\begin{figure}
  \center
  \includegraphics[width=5.0cm]{web_tree.eps}
  \caption{The application tree (an excerpt).}
  \label{Figure: web tree}
\end{figure}

So, the web application has a graph structure with a root node (an entry
point to the graph). The URI path goes from the root to another node in
the graph.

Once the node is reached, the given {\tt <method>} of the resource is called.
That's what the {\tt itools.web} publisher does, follow the path to reach
the node and call the method.

The use of the semicolon to separate the method from the path makes the
URI explicit, you look at it and you know what is the path, which nodes
in the graph the path goes through, and which method is called at the end.

Note that the character semicolon has been choosen because it is defined
by the RFC 2396\footnote{http://www.ietf.org/rfc/rfc2396.txt}, which defines
the URI standard. It is the character that, within a path segment, separates
the resource name from the segment parameters.



\section{The Context}

Wherever in the code there is available a global object named the
{\em context}, it contains:

\begin{api}
  {\tt request}\\
  - The request object.

  {\tt response}\\
  - The response object.

  {\tt user}\\
  - The authenticated user, or {\tt None} if it is an anonymous (non
    authenticated) user.

  {\tt root}\\
  - The handler for the root resource.

  {\tt handler}\\
  - The handler of the resource being published (usually it is {\tt self}).

  {\tt path}\\
  - The path from the root handler to the published handler (an instance
    of the {\tt itools.uri.Path} class).

  {\tt method}\\
  - The name of the method being published, or {\tt None} if the url did
    not specified the method.
\end{api}

The context can be accessed through the {\tt itools.web.get\_context}
method. A pattern that is very often found within the ikaaro's code is:

\begin{code}
    from itools.web import get_context

    context = get_context()
    request, response = context.request, context.response
\end{code}


\subsection{The Request}

The request object is a wrapper around the Zope's request object, it
provides a higher level API:

\begin{api}
    {\tt uri}\\
    - The requested uri (an {\tt itools.uri} reference).

    {\tt referer}\\
    - The referer uri (an {\tt itools.uri} reference), or {\tt None} if
      there is not a referer.

    {\tt accept\_language}\\
    - An instance of the {\tt itools.i18n.accept.AcceptLanguage} class, it
      keeps the user preferred languages.

    {\tt form}\\
    - The request parameters passed either through the query or as form
      values. It is mapping from key to value.

    {\tt cookies}\\
    - The cookies, it is a mapping.
\end{api}


\subsection{The Response}

The response object is a wrapper around the Zope's response object, it
provides a higher level API:


\begin{api}
  {\tt has\_header(name)}\\
  - 

  {\tt get\_header(name)}\\
  - 

  {\tt set\_header(value)}\\
  - 

  {\tt del\_cookie(name)}\\
  - 

  {\tt set\_cookie(name, value, **kw)}\\
  - 

  {\tt redirect(uri)}\\
  - 

  {\tt set\_status(status)}\\
  - 
\end{api}