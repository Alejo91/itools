\chapter{Addendum}

This documentation is work in progress. It has covered only about half the
packages provided by {\tt itools}. It remains to look at the workflow engine
from {\tt itools.workflow} and the index and search possibilities offered by
{\tt itools.catalog}.

And the topics we have talked about have only been exposed on the surface.
The API provided by the different classes is actually bigger than what this
document shows. The examples are minimal, and leave most possibilities to
the reader's imagination. We haven't seen all the available handlers in detail.

Anyway, I hope you have found these tools interesting enough to get a closer
look.

If you didn't, I have still a few words to add.


\section{In the real world}

In spite of the youth of {\tt itools} (version {\tt 0.5} at the time of this
writing), it is already being used in real production projects. Specially,
it is the foundation of the {\bf iKaaro}\footnote{http://www.ikaaro.org}
Content Management System.

Some of the advantages it provides include:

\begin{itemize}
  \item The whole web site data is stored as a tree of human readable
    files, instead of opaque persistent Python objects. It is possible
    to export a web site to the file system, to inspect and manipulate
    the tree, to make a tarball, to import it back to the ZODB.

  \item This approach encourages the use of standards, every time we
    need a new feature we look first if there is already a standard
    file format. The result is a highly portable, standards compliant
    application, easy to interact with other applications.

  \item It is possible to seamlessly integrate content from foreign
    sources and to make it appear as if they were in our web site.
    We are just at the surface of the possibilities.
\end{itemize}

But the most important, it provides a simple, high-productive framework
to build our applications on, to reduce the development time to the
minimum.

\section{Future works}

Some of the stuff waiting on the pipe\ldots

\begin{itemize}
  \item Schemas. Not only for XML documents (XML
    Schema\footnote{http://www.w3.org/XML/Schema} support is on the works),
    but also other handlers, specially folders.

  \item Many handlers for standard file formats: iCal, vCard, TMX, XLIFF, etc.

  \item Make {\tt itools.uri} 100\% compliant with the RFC2396,
    maybe write a {\bf P}ython {\bf E}xtension {\bf P}roposal to
    get it in the Standard Library.

  \item Support for new schemes, protocols and storages for
    {\tt itools.resources}.

  \item Add the last bricks to {\tt itools.i18n} to get an engine able
    to power high end translation memory systems.
\end{itemize}


\section{Free Software}

And, last but not least, {\tt itools} is free software, released to the
world under the terms and conditions of the GNU Lesser General Public
License\footnote{http://www.gnu.org/copyleft/lesser.html}, developers
own the copyright of the code they write, contributions are very welcomed.
