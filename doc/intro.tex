\chapter{Introduction}

This paper is the official documentation of {\tt itools}.

\section{What is {\tt itools}?}

{\tt itools} is a Python\footnote{\tt http://www.python.org} package.
Originally it was intended to develop web applications, but being low
level stuff it is likely to be found useful in other contexts.

It consists of a collection of sub-packages and modules which makes up a
coherent set of tools for the developer. A summary of these tools follows:

\begin{itemize}
 \item {\tt itools.uri} -- an API to manage URIs, to identify and locate
    resources.

 \item {\tt itools.resources} -- an abstraction layer over resources that
   lets to manage them with a consistent API, independently of where they
   are stored.

 \item {\tt itools.handlers} -- resource handlers infrastructure (resource
   handlers are non persistent classes that add specific semantics to
   resources). This package also includes several handlers {\em out of the
   box}.

 \item {\tt itools.xml} -- XML infrastructure, includes resource handlers
   for XML, XHTML and HTML documents. Plus the {\bf S}imple {\bf T}emplate
   {\bf L}anguage.

 \item {\tt itools.i18n} -- tools for language negotiation and text
   segmentation.

 \item {\tt itools.workflow} -- represent workflows as automatons, objects
   can move from one state to another through transitions, classes can add
   specific semantics to states and transitions.

 \item {\tt itools.catalog} -- An index and search engine.
\end{itemize}

One of the criterias behind the {\tt itools} architecture is to reduce the
dependencies between the different modules to the minumum, what makes up a
flexible set of tools. Think of it as an extension to the Python's Standard
Library.


\subsection{The resource-handler model}

Of everything in {\tt itools} I am probably most proud of the the three
sub-packages {\tt itools.uri}, {\tt itools.resources} and
{\tt itools.handlers}, which make up what I call the {\em resource-handler}
model.

There is a linear relationship of dependency between these modules, {\tt uri}
depends on nothing but Python, {\tt resources} depends on {\tt uri}, and
{\tt handlers} depends on {\tt resources}.

This allows to use these modules on a flexible way. For example, you can
just use {\tt itools.uri} (ignoring the others) to benefit from a higher
level API to work with URIs than that provided by the Standard Library
({\tt urlparse}).

Or you may use {\tt itools.resources} if you want to benefit from an
abstraction layer over the storage. This is to say, if you want to be able
to manipulate many resources stored in different systems and accessed
through different protocols with the same, consistent, rich API.

You may even take advantage of {\tt itools.handlers} without fully
understanding the model behind. For example, you could use some of the
handlers {\tt itools} offers out-of-the-box for standard file formats
like CSV, PO or XHTML, to simplify your live if you ever need to work
with one of these formats.

But, in order to exploit everything {\tt itools} has to offer to its limit,
you may choose to base part or all of your application architecture on the
{\em resource-handler} model, a model heavily influenced by the filesystem
and the Web. In this situation the three packages ({\tt uri}, {\tt resources}
and {\tt handlers}) show themselves as three different components with
a distinct role in the architecture:

\begin{itemize}
  \item {\tt itools.uri} identifies and locates resources, wherever they
    are. Altogether with {\tt itools.resources} it effectively enables your
    information system to be distributed through an heterogeneous base.

  \item {\tt itools.resources} provides persistency to your data, this is
    to say, resources are where the data is stored.

  \item and {\tt itools.handlers} is where the logic lives. The essential
    characteristic of a resource handler (or handler for short) is that it is
    non-persistent, instead it is associated with a resource, whose content
    it is responsible to manage.
\end{itemize}

The first chapters of this document will cover these sub-packages,
{\tt itools.uri}, {\tt itools.resources} and {\tt itools.handlers} with
detail.

\subsection{eXtensible Markup Language}

The {\tt itools.xml} depends on {\tt itools.handlers}, this is its first
advantage. It represents an XML document as an DOM like tree. But it
also provides support for schemas, what allows to easily build higher
level data structures.

The package also provides handlers for specific document types out-of-the-box,
most notably XHTML and HTML (even if HTML is not XML, it shares a lot with
XHTML).


\subsection{The Simple Template Language}

The Simple Template Language is included in the {\tt itools.xml} package,
but it is important enough to deserve its own chapter in the documentation.

Its design goals are:

\begin{itemize}
  \item Truly separate logic and presentation. Even Python expressions are
    not allowed within the template.

  \item Really simple. It can be mastered in half a day.

  \item Damn fast (easy to achieve through simplicity).

  \item Secure. Even non-trusted users could write templates without risk,
    because code is not allowed within the template.
\end{itemize}

And the key idea behind, is to make it a {\em descriptive} language.


\subsection{Workflow}

The sub-package {\tt itools.workflow} (once known as {\em flux}) is the
oldest code in {\tt itools}. It does not depend on anything but the
Standard Library, and don't puts any restriction on the storage. This
makes it very easy to combine with other frameworks.

There is a chapter exclusively dedicated to it.

\subsection{Internationalization and Localization}

The {\tt itools.i18n} package provides a wide range of tools for
internationalization and localization of both software and data. From
message extraction to language negotiation, through text segmentation,
fuzzy matching or algorithms to guess the language a text is written in.

A chapter is devoted to this topic.


\subsection{Index and Search}

The {\tt itools.catalog} sub-package provides an index and search engine.
Though still young, it already provides full text indexing, boolean queries
and results sorted by weight.



\section{Who is this document for?}

This document is addressed to Python developers. Though it may be useful
to software architects in general, as some ideas exposed here could be
found interesting\footnote{For example, {\tt itools.workflow} inspired
XXX to write a similar engine in Java, see {\tt http://XXX}}.

It is convenient to have some basic skills with the Python programming
language to fully understand this document. Probably the best introduction
to Python is the official tutorial:

\begin{quote}
  {\tt http://python.org/doc/2.3.4/tut/tut.html}.
\end{quote}

This document touches many different technologies and standards, such as XML.
References will be given in the relevant chapters.


\section{Project status}

Currently {\tt itools} is in alpha stage of development (at the time of
this writing the last version available is {\bf 0.5}). This means that the
API is not yet stable, the documentation is incomplete, and there may be bugs.


\section{Installation}

Before going further, be sure your system is correctly setup. First you
will need Python 2.3.4 or later.

Then download the last {\tt itools} version from {\tt http://www.ikaaro.org},
unpack it somewhere, and install it; {\tt itools} uses {\em distutils}, so
just type:

\begin{code}
    $ python setup.py install
\end{code}

Be sure to have the right permissions.


\section{About this document}

The development keeps ahead of the documentation. I would say the docs cover
around 60\% of {\tt itools}.

There are also two appendixes, one explaining the coding style {\tt itools}
is written in, another one introducing the use of {\em GNU arch}. Both are
specially addressed to those that want to contribute back to the main tree.
