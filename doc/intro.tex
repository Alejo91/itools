\chapter{Introduction}

This paper is the official documentation of {\tt itools}.

\section{What is {\tt itools}?}

{\tt itools} is a Python\footnote{\tt http://www.python.org} package.
Originally it was intended to develop web applications, but being low
level stuff it is likely to be found useful in other contexts.

It consists of a collection of sub-packages and modules which makes up a
coherent set of tools for the developer. A summary of these tools follows:

\begin{itemize}
 \item {\tt itools.uri} -- an API to manage URIs, to identify and locate
    resources.

 \item {\tt itools.resources} -- an abstraction layer over resources that
   lets to manage them with a consistent API, independently of where they
   are stored.

 \item {\tt itools.handlers} -- resource handlers infrastructure (resource
   handlers are non persistent classes that add specific semantics to
   resources). This package also includes several handlers {\em out of the
   box}.

 \item {\tt itools.xml} -- XML infrastructure, includes resource handlers
   for XML, XHTML and HTML documents. Plus the {\bf S}imple {\bf T}emplate
   {\bf L}anguage.

 \item {\tt itools.i18n} -- tools for language negotiation and text
   segmentation.

 \item {\tt itools.workflow} -- represent workflows as automatons, objects
   can move from one state to another through transitions, classes can add
   specific semantics to states and transitions.

 \item {\tt itools.lucene} -- An underway implementation of the famous
   Jakarta's indexing and search tool:
   {\em Lucene}\footnote{\tt http://jakarta.apache.org/lucene}.
\end{itemize}



\section{Project status}

Currently {\tt itools} is in alpha stage of development (at the time of
this writing the last version available is {\bf 0.4}). This means that the
API is unstable, the documentation is incomplete, and there are many bugs.
You are warned.

\section{About this document}

The development keeps ahead of the documentation. Today only few aspects
of {\tt itools} are documented. By order of appearance:

\begin{enumerate}
  \item The first chapter is dedicated to {\tt itools.uri}: we will learn how
    to manipulate {\bf U}niform {\bf R}esource {\bf I}dentifiers.

  \item Follows {\tt itools.resources}, the file like abstraction layer.

  \item In third place comes {\tt itools.handlers}, or the art of working
    with resources.

  \item Next {\tt itools.xml} is explained, the XML services provided by
    {\tt itools}, which are based on resource handlers.

  \item Closing is the {\bf S}imple {\bf T}emplate {\bf L}anguage, as unknown
    by the Python community as it is simple, powerful and fast.

  \item As an appendix the use of GNU arch will be quickly exposed, what
    will allow users to keep track of the {\tt itools} development.
\end{enumerate}


\subsection{Background}

Remember that this is a Python package, if you don't know about the Python
programming language better go read the tutorial first:

% XXX
\begin{quote}
  {\tt http://python.org/doc/2.3.4/tut/tut.html}.
\end{quote}

Through this document different technologies will be exposed. Proper
references will be given.

\section{Get ready}

Before going further, be sure your system is correctly setup. First you
will need Python 2.3 or later.

Then download the last {\tt itools} version from
{\tt http://sf.net/projects/lleu}, unpack it somewhere, and install it;
{\tt itools} uses {\em distutils}, so just type:

\begin{code}
    $ python setup.py install
\end{code}

Be sure to have the right permissions.

