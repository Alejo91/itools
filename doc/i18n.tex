\chapter{Internationalization and Localization}
\label{Chapter: i18n}

Or the art of developing multilingual applications. Let's start with the
concepts\footnote{Definitions take fron the GNU gettext documentation.}:

\begin{description}
  \item [Internationalization] the operation by which a program, or a set
    of programs turned into a package, is made aware of and able to support
    multiple languages. This is a generalization process, by which the
    programs are untied from calling only English strings or other English
    specific habits, and connected to generic ways of doing the same, instead.

  \item [Localization] the operation by which, in a set of programs already
    internationalized, one gives the program all needed information so that
    it can adapt itself to handle its input and output in a fashion which is
    correct for some native language and cultural habits. This is a
    particularisation process, by which generic methods already implemented
    in an internationalized program are used in specific ways. 
\end{description}

We are going to illustrate both processes, in the context of {\tt itools},
by extending the addressbook example.


\section{Internationalization}

What must be internationalized is the user interface, what in our example
is reduced to the template {\tt addressbook\_view.xml}.

\subsection{Naming conventions}

The technique we use with {\tt itools} requires to have one template per
language, being english the master language. Each template must have a
distinct name, we will achieve it by appending the language code to the
end of the file:

\begin{code}
    addressbook_view.xml.en
    addressbook_view.xml.es
    addressbook_view.xml.fr
    ...
\end{code}

So the first thing to do is to rename the file {\tt addressbook\_view.xml}
to {\tt addressbook\_view.xml.en}.

Note that only the template in English belongs to the source, the others
will be automatically built with the help of {\tt itools}.

\subsection{Template selection}

The method {\tt view} also must be modified, so the right template is used:

\begin{code}
    def view(self, language='en'):
        # Load the STL template
        resource_name = 'addressbook_view.xml.%s' % language
        resource = get_resource(resource_name)
        template = XML.Document(resource)

        ...
\end{code}


What remains to explain is how the application chooses the language the
user wants to see the addressbook in. This topic is out of the scope
of this chapter.

\section{Localization}

To illustrate the example we will use Spanish as the second language.

The issue is, from the template {\tt addressbook\_view.xml.en} we
must produce a template named {\tt addressbook\_view.xml.es}, where
the translatable messages will be in Spanish instead of English.

In our case there is only one translatable message, the word
{\tt Addressbook}. While the English template contains the header:

\begin{code}
    <h3>Addressbook</h3>
\end{code}

the Spansih one must have:

\begin{code}
    <h3>Libro de direcciones</h3>
\end{code}

The localization process is splitted in three stages:

\begin{enumerate}
  \item Message extraction.
  \item Human translation.
  \item Template re-construction.
\end{enumerate}

Through the whole process we will use {\tt GNU
gettext}\footnote{http://www.gnu.org/software/gettext/}, which is the
standard in the Free Software world for software internationalization
and localization; and the script {\tt igettext.py} provided by {\tt itools}.

The Figure~\ref{Figure: i18n} shows the localization process.

\begin{figure}
  \center
  \includegraphics[width=\textwidth]{i18n.eps}
  \caption{The localization process}
  \label{Figure: i18n}
\end{figure}

\subsection{Message extraction}

This step consists on the parsing of the templates that define the user
interface (in our case the file {\tt addressbook\_view.xml.en}), the
extraction of the translatable messages they contain, and the building of
a {\em PO} file which will be the input for the human translator.

This all is done with the {\tt igettext.py} script:

\begin{code}
    $ python igettext.py --pot addressbook_view.xml.en > es.po
\end{code}

The result is the {\tt es.po} file, whose content will be:

\begin{code}
    msgid ""
    msgstr ""
    "Project-Id-Version: PACKAGE VERSION\n"
    "POT-Creation-Date: 2004-10-28 17:10+CET\n"
    "PO-Revision-Date: YEAR-MO-DA HO:MI+ZONE\n"
    "Last-Translator: FULL NAME <EMAIL@ADDRESS>\n"
    "Language-Team: LANGUAGE <LL@li.org>\n"
    "MIME-Version: 1.0\n"
    "Content-Type: text/plain; charset=UTF-8\n"
    "Content-Transfer-Encoding: 8bit\n"

    #: addressbook_view.xml.en:0
    msgid "Addressbook"
    msgstr ""
\end{code}

The file consist of a header and the translatable messages that have been
found, in our case just the word {\tt Addressbook}. This file is the one
the translator will work with.


\section{Human translation}

The human translator receives the PO file, and must return the same PO
file, but including the translations for every message. In our example
he should return the file (excluding the header):

\begin{code}
    #: addressbook_view.xml.en:0
    msgid "Addressbook"
    msgstr "Libro de direcciones"
\end{code}

To do his work he may use a simple text editor like {\em vi} or {\em emacs}.
But typically he will use one of the PO graphic editors available, for
example:

\begin{itemize}
  \item KBabel, http://www.kde.org/
  \item poedit, http://poedit.sourceforge.net/
\end{itemize}


\section{Template building}

To close the process, we must generate the template
{\tt addressbook\_view.xml.es} from the master template
({\tt addressbook\_view.xml.en}) and from the translated PO file ({\tt es.po}).
This is done with the script {\tt igettext.py}:

\begin{code}
    $ igettext.py --xhtml addressbook_view.xml.en es.po \
          > addressbook_view.xml.es
\end{code}

And that's it, now we have the two templates, one in English and one in
Spanish.


